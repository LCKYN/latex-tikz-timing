\documentclass{standalone}
% Start of timing.sty

% Some macros for logic timing diagrams.
%
% Author: ir. Pascal T. Wolkotte and Jochem Rutgers, University of Twente
% Version: 0.1
% Date: 2007/10/11

\usepackage{tikz}

\newcounter{wavenum}

\setlength{\unitlength}{1cm}
% advance clock one cycle, not to be called directly
\newcommand*{\clki}{
  \draw (t_cur) -- ++(0,.3) -- ++(.5,0) -- ++(0,-.6) -- ++(.5,0) -- ++(0,.3)
    node[time] (t_cur) {};
}

\newcommand*{\bitvector}[3]{
  \draw[fill=#3] (t_cur) -- ++( .1, .3) -- ++(#2-.2,0) -- ++(.1, -.3)
                         -- ++(-.1,-.3) -- ++(.2-#2,0) -- cycle;
  \path (t_cur) -- node[anchor=mid] {#1} ++(#2,0) node[time] (t_cur) {};
}

% \known{val}{length}
\newcommand*{\known}[2]{
    \bitvector{#1}{#2}{white}
}

% \unknown{length}
\newcommand*{\unknown}[2][XXX]{
    \bitvector{#1}{#2}{black!20}
}

% \bit{1 or 0}{length}
\newcommand*{\bit}[2]{
  \draw (t_cur) -- ++(0,.6*#1-.3) -- ++(#2,0) -- ++(0,.3-.6*#1)
    node[time] (t_cur) {};
}

% \unknownbit{length}
\newcommand*{\unknownbit}[1]{
  \draw[ultra thick,black!50] (t_cur) -- ++(#1,0) node[time] (t_cur) {};
}

% \nextwave{name}
\newcommand{\nextwave}[1]{
  \path (0,\value{wavenum}) node[left] {#1} node[time] (t_cur) {};
  \addtocounter{wavenum}{-1}
}

% \clk{name}{period}
\newcommand{\clk}[2]{
    \nextwave{#1}
    \FPeval{\res}{(\wavewidth+1)/#2}
    \FPeval{\reshalf}{#2/2}
    \foreach \t in {1,2,...,\res}{
        \bit{\reshalf}{1}
        \bit{\reshalf}{0}
    }
}

% \begin{wave}[clkname]{num_waves}{clock_cycles}
\newenvironment{wave}[3][clk]{
  \begin{tikzpicture}[draw=black, yscale=.7,xscale=1]
    \tikzstyle{time}=[coordinate]
    \setlength{\unitlength}{1cm}
    \def\wavewidth{#3}
    \setcounter{wavenum}{0}
    \nextwave{#1}
    \foreach \t in {0,1,...,\wavewidth}{
      \draw[dotted] (t_cur) +(0,.5) node[above] {t=\t} -- ++(0,.4-#2);
      \clki
    }
}{\end{tikzpicture}}

%%% End of timing.sty
\begin{document}
\begin{wave}{13}{5}
 \nextwave{req\_addr} \bit{0}{.2} \bit{1}{1} \bit{0}{3} \bit{1}{1} \bit{0}{.8}
 \nextwave{inst\_addr} \unknown[X]{.5} \known{addr}{1} \unknown{4.5}
 \nextwave{link\_addrs} \unknown{1.2} \known{map}{1} \unknown{3.8}
 \nextwave{link\_load} \unknown{2.2} \known{vam}{1} \unknown{2.8}
 \nextwave{link\_load\_r} \unknown{3.1} \known{val}{1} \unknown{1.9}
 \nextwave{simulate} \bit{0}{3.1} \bit{1}{1} \bit{0}{1.9}
 \nextwave{output} \unknown{3.3} \known{}{.5} \known{}{.5} \unknown{1.7}
 \nextwave{prev\_output} \unknown{3.2} \known{old}{1} \unknown{1.8}
 \nextwave{differs} \unknownbit{3.3} \bit{1}{.5} \bit{1}{.4} \unknownbit{1.8}
 \nextwave{differs\_r} \unknownbit{4.1} \bit{1}{1} \unknownbit{.9}
 \nextwave{dep\_addr} \unknown{4.2} \known{dep}{1} \unknown[X]{.8}
 \nextwave{req} \unknown{4.3} \known{req}{1} \unknown[X]{.7}
\end{wave}

\end{document}